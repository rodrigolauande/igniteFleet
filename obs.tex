=> instalação e configurações do projeto padrão

npm install styled-components
npm install --save-dev @types/styled-components
npm install --save-dev @types/styled-components-react-native

npx expo install expo-font @expo-google-fonts/roboto

=> definição da 1º pagina no arquivo App.json

{
    "expo": {
      "name": "igniteFleet",
      "slug": "igniteFleet",
      "version": "1.0.0",
      "orientation": "portrait",
      "icon": "./assets/icon.png",
      "userInterfaceStyle": "dark",
      "splash": {
        "image": "./assets/splash.png",
        "resizeMode": "cover",
        "backgroundColor": "#202024"
      },
      "assetBundlePatterns": [
        "**/*"
      ],
      "ios": {
        "supportsTablet": true
      },
      "android": {
        "adaptiveIcon": {
          "foregroundImage": "./assets/adaptive-icon.png",
          "backgroundColor": "#202024"
        }
      },
      "web": {
        "favicon": "./assets/favicon.png"
      },
      "plugins": [
        "expo-font"
      ]
    }
  }

=> definição do
  theme,
  components/Loading,
  @types/png.d.ts
  @types/styled.d.ts
  screens/Signin/index e styles,
  components/Button,


=> OAUTH 2.0 (Autenciação do usuario)
- react-native-dotenv (cuidando das variaveis de ambiente)
- npm install -D react-native-dotenv
configurar o dotenv na aplicação no arquivo babelConfig.config.js

module.exports = function(api) {
  api.cache(true);
  return {
    presets: ['babel-preset-expo'],
    plugins: [
      [
        'module:react-native-dotenv',
        {
          'moduleName': '@env',
          'allowUndefined': false,
        }
      ]
    ]
  };
};


-definição de uma typagem das variaveis de ambiente

declare module '@env' {
  export const ANDROID_CLIENT_ID: string;
  export const IOS_CLIENT_ID: string
}

- cria as variaveis de ambiente .env raiz do projeto

ANDROID_CLIENT_ID=''
IOS_CLIENT_ID=''

-import na app.tsx (principal)

import { ANDROID_CLIENT_ID } from '@env';

=> CRIA O PROJETO NO ICLOUD DA GOOGLE
-CRIAR AS CHAVES PRIVADAS E PUBLICAS

=> muda o projeto para ter acesso as pasta 'android e ios'
npx expo prebuild
ele pergunta o nome do pacote 'com.lauande.ignitefleet'

cria a build do projeto novamente instalando as novas configurações
npx expo run:android

dentro da pasta android digite o comando


